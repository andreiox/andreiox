\documentclass[11pt,a4paper,sans]{moderncv}

\moderncvstyle{classic}
\moderncvcolor{orange}

\renewcommand*{\mobilephonesymbol}   {{\Large\faMobilePhone}~}
\renewcommand*{\emailsymbol}         {{\small\faInbox}~}

\usepackage[utf8]{inputenc}
\usepackage[scale=0.85]{geometry}
\usepackage{fontawesome}

\name{André}{Silva Macedo}
\title{\normalfont 28 anos, Brasileiro}
\address{\normalfont Belo Horizonte, Brasil}
\phone[mobile]{\normalfont +55~(31)~99838-5766}
\email{contato@andremacedo.dev}
\extrainfo{\normalfont\faGithub{ }\httplink{github.com/andreiox}}

\begin{document}
\makecvtitle

\section{Educação}
\cventry{2016--2019}{Bacharelado}{Universidade FUMEC}{Belo Horizonte - Brasil}{\textit{Ciência da Computação}}{}

\section{Sobre mim}
\cvitem{}{
    Um desenvolvedor full-stack dedicado com 8 anos de experiência em desenvolvimento de software.
    Poliglota se tratando em linguagens de programação em Java, Node.js, Python e, mais recentemente, Clojure.
    Experiência com desenvolvimento de software de um novo produto, da ideia à entrega.
    Bem como experiência em manutenção, refatoração e adição de funcionalidades a um software existente.
}

\section{Experiência}
\cventry{2022--agora}{Engenheiro de Software Sênior}{Nubank}{}{}{
    Trabalho na área de investimentos com produtos da B3 usando tecnologias relacionadas ao Clojure.
}
\cventry{2020--2022}{Team Leader}{Seidor Véritas}{}{}{
    Lider da equipe responsável pela manutenção e evolução do módulo de emissão de documentos fiscais (Outbound) da plataforma Orbit by Seidor.
}
\cventry{2019--2020}{Desenvolvedor Back-end}{Seidor Véritas}{}{}{
    Desenhei e implementei micro-serviços REST e AWS Lambdas usando Node.js e Java 8.
    Desenvolvimento bastante focado em testes automatizados (testes de unidade e integração) e documentação da API já que a API é utilizada por clientes externos;
    Responsável pelo deploy de toda infraestrutura utilizando Infraestrutura como Código (aws-cdk).
    Infra composta de Elastic Beanstalk, ECS, Fargate, Lambda, API Gateway, e mais.
}
\cventry{2018--2019}{Desenvolvedor Full-stack}{Melhor Voo}{}{}{
    Desenvolvi várias páginas web e componentes usando Node.js, React.js, Redux e JSS;
    Implementei APIs RESTful usando Express.js, Laravel e Django Rest Framework;
    Trabalhei com bancos de dados MySQL e MongoDB;
    Desenvolvi web crawlers usando Node.js e Python.
}
\cventry{2016--2018}{Desenvolvedor Java Full-stack}{Registrocom.com}{}{}{
    Desenvolvi em grande parte aplicações web usando Java 7, JSF 2.2 e 2, Primefaces 6 e 3, JPA (Hibernate e Eclipselink), MySQL e Apache Tomcat 8.
}

\section{Knowledge}
\cvitem{Frontend}{React, Primefaces, Flutter}
\cvitem{Backend}{Node.js, Java, Clojure, Python}
\cvitem{Frameworks}{Express, Fastify, Django, Flask, Spring Boot}
\cvitem{Databases}{MySQL, Postgres, MongoDB, DynamoDB, Datomic}
\cvitem{AWS}{Lambda, Step Functions, ECS, Fargate, EC2, Elastic Beanstalk, S3, API Gateway, Lambda Authorizers, RDS, Cognito, SQS, CodeBuild, CodePipeline, CDK (infra as code)}
\cvitem{Other}{Linux, Git, Scrum, Automated Tests}

\section{Idiomas}
\cvitemwithcomment{Português}{Nativo}{}
\cvitemwithcomment{Inglês}{Avançado}{}

\end{document}
